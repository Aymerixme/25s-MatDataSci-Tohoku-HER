\documentclass{article}

% Language setting
% Replace `english' with e.g. `spanish' to change the document language
\usepackage[english]{babel}

% Set page size and margins
% Replace `letterpaper' with `a4paper' for UK/EU standard size
\usepackage[letterpaper,top=2cm,bottom=2cm,left=3cm,right=3cm,marginparwidth=1.75cm]{geometry}

% Useful packages
\usepackage{amsmath}
\usepackage{graphicx}
\usepackage[colorlinks=true, allcolors=blue]{hyperref}

\usepackage{comment}

\title{Deep Learning for Properties prediction based on 3D properties with }
%\author{Yuto Tsuruta}
\author{Aymeric Hernandez\\
  \and
  Nay Chi Hnin Htut\\
  \and
  Yuto Tsuruta
}


\begin{document}
\maketitle

\begin{abstract}
The accurate prediction of molecular properties from structural information is essential for accelerating discovery in chemistry and materials science. While Density Functional Theory (DFT) provides reliable quantum mechanical predictions, its high computational cost limits its applicability in large-scale screening. In this study, we develop a neural network-based regression model to predict molecular properties—specifically total energy—directly from three-dimensional atomic coordinates. Using the DFT\_all.npz dataset available from Zenodo, which contains a variety of DFT-computed properties for small organic molecules, we train the model in a supervised manner to learn the structure–property relationship. Our results demonstrate that neural networks can effectively approximate DFT-level accuracy while significantly reducing computation time. This work highlights the potential of machine learning as a scalable alternative to traditional quantum chemical simulations, enabling faster exploration of chemical space for materials and drug design.
\end{abstract}

\section{Introduction}

Predicting molecular properties directly from structural information is a fundamental task in computational chemistry and materials science. Traditionally, this is achieved through quantum mechanical methods such as Density Functional Theory (DFT), which provide accurate predictions but are computationally expensive and limited in scalability. As the demand grows for rapid property evaluation in high-throughput screening and molecular design, data-driven alternatives have gained significant attention.



Recent advances in machine learning, particularly neural networks, have opened new pathways for modeling the complex relationship between a molecule’s structure and its physicochemical properties. These models can learn from large datasets of precomputed molecular structures and properties to make fast, accurate predictions without relying on costly simulations.



In this project, we focus on \textbf{predicting molecular properties from 3D molecular structures} using supervised learning with neural networks. We use the \textbf{QM24} dataset, derived from DFT calculations and available through [Zenodo](https://zenodo.org/records/11164951), which contains atomic coordinates and quantum-level properties for a variety of small organic molecules.



Our goal is to \textbf{train a neural network to accurately predict key molecular properties—such as total energy—from 3D atomic coordinates}, thereby capturing the structure–property relationship encoded in quantum mechanical simulations. This approach aims to demonstrate how machine learning models can serve as efficient surrogates for DFT, accelerating materials discovery and molecular design through predictive modeling.


\section{Data Science Method}

We decided to realize a Property prediction of the elements from their 3D molecular structure. We will use Supervising training on Neural Network. To do so we will use the 3D properties of the molecules as training inputs, then we will train the network with the use of the desired Property (Atomization Energy for example) as a label.

The input data should need no pretreatment. The output label should be a continuous value defining the property of the element.

The dataset is composed of 784875 element which is a quite huge amount of data (to compare, MNIST which is a basic digit recognition dataset contains 70000 elements) so the split between training subset and test subset should be relevant.

The main limit of this method is for each property we would like to predict, we would have to entirely redo the training with a different label.  

\section{Exploratory Data Analysis}
    \subsection{Explanation of our data set}

\begin{center}
    \begin{tabular}{||c c||} 
        \hline
            Variable name & Python representation format\\
        \hline\hline
        compounds & array \\
        \hline
   atoms & array\\
   \hline
   freqs & array\\
   \hline
   vibmodes & array\\
   \hline
   zpves & float64\\
   \hline
   U0 & float64\\
   \hline
   U298 & float64\\
   \hline
   H & float64\\
   \hline
   S & float64\\
   \hline
   G & float64\\
   \hline
   Cv & float64\\
   \hline
   Cp & float64\\
   \hline
   coordinates & array\\
   \hline
   Vesp & array\\
   \hline
   Qmulliken & array\\
   \hline
   dipole & array\\
   \hline
   quadrupole & array\\
   \hline
   octupole & array\\
   \hline
   hexadecapole & array\\
   \hline
   rots & array\\
   \hline
   gap & float64\\
   \hline
   Eee & float64\\
   \hline
   Exc & float64\\
   \hline
   Edisp & float64\\
   \hline
   Etot & float64\\
   \hline
   Eatomization & float64\\
   \hline

    \end{tabular}
\end{center}





\begin{comment}
\subsection{How to include Figures}

First you have to upload the image file from your computer using the upload link in the file-tree menu. Then use the includegraphics command to include it in your document. Use the figure environment and the caption command to add a number and a caption to your figure. See the code for Figure \ref{fig:frog} in this section for an example.

Note that your figure will automatically be placed in the most appropriate place for it, given the surrounding text and taking into account other figures or tables that may be close by. You can find out more about adding images to your documents in this help article on \href{https://www.overleaf.com/learn/how-to/Including_images_on_Overleaf}{including images on Overleaf}.


\begin{figure}
\centering
\includegraphics[width=0.25\linewidth]{frog.jpg}
\caption{\label{fig:frog}This frog was uploaded via the file-tree menu.}
\end{figure}

\subsection{How to add Tables}

Use the table and tabular environments for basic tables --- see Table~\ref{tab:widgets}, for example. For more information, please see this help article on \href{https://www.overleaf.com/learn/latex/tables}{tables}. 

\begin{table}
\centering
\begin{tabular}{l|r}
Item & Quantity \\\hline
Widgets & 42 \\
Gadgets & 13
\end{tabular}
\caption{\label{tab:widgets}An example table.}
\end{table}

\subsection{How to add Comments and Track Changes}

Comments can be added to your project by highlighting some text and clicking ``Add comment'' in the top right of the editor pane. To view existing comments, click on the Review menu in the toolbar above. To reply to a comment, click on the Reply button in the lower right corner of the comment. You can close the Review pane by clicking its name on the toolbar when you're done reviewing for the time being.

Track changes are available on all our \href{https://www.overleaf.com/user/subscription/plans}{premium plans}, and can be toggled on or off using the option at the top of the Review pane. Track changes allow you to keep track of every change made to the document, along with the person making the change. 

\subsection{How to add Lists}

You can make lists with automatic numbering \dots

\begin{enumerate}
\item Like this,
\item and like this.
\end{enumerate}
\dots or bullet points \dots
\begin{itemize}
\item Like this,
\item and like this.
\end{itemize}

\subsection{How to write Mathematics}

\LaTeX{} is great at typesetting mathematics. Let $X_1, X_2, \ldots, X_n$ be a sequence of independent and identically distributed random variables with $\text{E}[X_i] = \mu$ and $\text{Var}[X_i] = \sigma^2 < \infty$, and let
\[S_n = \frac{X_1 + X_2 + \cdots + X_n}{n}
      = \frac{1}{n}\sum_{i}^{n} X_i\]
denote their mean. Then as $n$ approaches infinity, the random variables $\sqrt{n}(S_n - \mu)$ converge in distribution to a normal $\mathcal{N}(0, \sigma^2)$.


\subsection{How to change the margins and paper size}

Usually the template you're using will have the page margins and paper size set correctly for that use-case. For example, if you're using a journal article template provided by the journal publisher, that template will be formatted according to their requirements. In these cases, it's best not to alter the margins directly.

If however you're using a more general template, such as this one, and would like to alter the margins, a common way to do so is via the geometry package. You can find the geometry package loaded in the preamble at the top of this example file, and if you'd like to learn more about how to adjust the settings, please visit this help article on \href{https://www.overleaf.com/learn/latex/page_size_and_margins}{page size and margins}.

\subsection{How to change the document language and spell check settings}

Overleaf supports many different languages, including multiple different languages within one document. 

To configure the document language, simply edit the option provided to the babel package in the preamble at the top of this example project. To learn more about the different options, please visit this help article on \href{https://www.overleaf.com/learn/latex/International_language_support}{international language support}.

To change the spell check language, simply open the Overleaf menu at the top left of the editor window, scroll down to the spell check setting, and adjust accordingly.

\subsection{How to add Citations and a References List}

You can simply upload a \verb|.bib| file containing your BibTeX entries, created with a tool such as JabRef. You can then cite entries from it, like this: \cite{greenwade93}. Just remember to specify a bibliography style, as well as the filename of the \verb|.bib|. You can find a \href{https://www.overleaf.com/help/97-how-to-include-a-bibliography-using-bibtex}{video tutorial here} to learn more about BibTeX.

If you have an \href{https://www.overleaf.com/user/subscription/plans}{upgraded account}, you can also import your Mendeley or Zotero library directly as a \verb|.bib| file, via the upload menu in the file-tree.

\subsection{Good luck!}

We hope you find Overleaf useful, and do take a look at our \href{https://www.overleaf.com/learn}{help library} for more tutorials and user guides! Please also let us know if you have any feedback using the Contact Us link at the bottom of the Overleaf menu --- or use the contact form at \url{https://www.overleaf.com/contact}.

\bibliographystyle{alpha}
\bibliography{sample}

\end{comment}

\end{document}

